\documentclass[a4paper]{article}
\usepackage[dutch]{babel}
\usepackage{microtype}
\usepackage{todonotes}
\usepackage{url}

\setlength{\voffset}{0pt} \setlength{\topmargin}{0pt}
\setlength{\headheight}{0pt} \setlength{\headsep}{0pt}
\setlength{\oddsidemargin}{2pt} \setlength{\evensidemargin}{2pt}
\setlength{\textwidth}{6in} \setlength{\textheight}{9.5in}

\def\Size{4pt}
\tikzset{
      folder/.pic={
        \filldraw[draw=folderborder,top color=folderbg!50,bottom color=folderbg]
          (-1.05*\Size,0.2\Size+5pt) rectangle ++(.75*\Size,-0.2\Size-5pt);
        \filldraw[draw=folderborder,top color=folderbg!50,bottom color=folderbg]
          (-1.15*\Size,-\Size) rectangle (1.15*\Size,\Size);
      }
    }

\title{Informaticawerktuigen - Groepsopdracht}

\begin{document}
\begin{center}
  \huge Informaticawerktuigen \\
  \Huge Groepsopdracht -- Opgave 4 \\
  \huge Presenteren
\end{center}
\vspace{1em}

Het geven van een presentatie is meestal geen eenvoudige opgave.
Niet alleen moet de inhoud interessant zijn, maar moet deze ook goed gebracht worden.
Daarnaast krijg je soms ook te maken met podiumvrees en tijdsdruk.
Dit document geeft je een aantal tips om een goede presentatie te maken en te geven.


\section{Deliverables}

De presentatie zullen jullie maken met het \LaTeX{}-pakket Beamer, waarover jullie een oefenzitting hebben gekregen.
Deze slides zullen jullie uiteindelijk gebruiken om de presentatie te geven tijdens \'e\'en van de aangeduide presentatiedagen (zie de samenvatting van deadlines) aan professor Berbers en jullie medestudenten.
Via Toledo laten we binnenkort weten welke groepjes op welke dag zullen moeten presenteren.

Omdat met de hele groep presenteren wat veel is, zullen jullie de groep in twee verdelen.
Van beide groepen wordt verwacht dat ze de presentatie kunnen geven.
Op de dag van de presentatie wordt \'e\'en van de twee groepen willekeurig geselecteerd, deze groep zal de presentatie voordragen.


\section{Evaluatiecriteria}

\subsection{De slides}

Een goede presentatie start natuurlijk bij goede slides.
De ervaring leert ons echter dat het maken van goede slides makkelijker lijkt dan het in feite is.
In deze sectie vind je enkele tips die je verder helpen bij het ontwerpen van de slides.


\paragraph{Eenvoud}

Het belangrijkste kenmerk van goede slides is eenvoud.
De inhoud van een slide is een \textit{samenvatting} van wat de spreker vertelt, niet het transcript.
Veel tekst en volzinnen horen niet thuis in een slide.
De mensen in het publiek zullen meestal eerst lezen wat er op de slides staat alvorens hun aandacht terug te richten op de spreker.
Indien de spreker telkens enkel vertelt wat er reeds op de slides te zien is, zal die al snel de aandacht van het publiek verliezen.
Een typische uitzondering op deze regel is natuurlijk in het geval dat een citaat getoond wordt.
Dit kan natuurlijk wel in zijn volledigheid overgenomen worden in de slide.

Het gebruik van een goede figuur of afbeelding komt de presentatie zeker ten goede.
Niet alleen zal hierdoor de aandacht van het publiek naar de spreker gaan (die hier dan natuurlijk een uitleg over geeft), maar het kan ook een complexer concept verduidelijken.


\paragraph{Taal en vormgeving}

Zorg ervoor dat, net zoals in je tekst, de taal in je presentatie verzorgd is.
Taalfouten in de slides geven natuurlijk geen goede indruk!
Zorg voor een consistent gebruik van hoofdletters.
Slides waar de verschillende puntjes soms wel en soms niet met hoofdletters beginnen komen nogal slordig over.
Je mag ook gekleurde, cursieve en vette tekst gebruiken, maar zorg er voor dat dit niet storend begint te werken.
Aangezien de inhoud van je slides sowieso al een samenvatting is van de meest belangrijke punten, zal het dan ook niet vaak voorkomen dat je bepaalde onderdelen nog eens apart moet gaan aanduiden.


\paragraph{Inhoud}

De inhoud van je presentatie moet uiteraard overeenkomen met de inhoud van je tekst, maar dat wil niet zeggen dat een presentatie louter een verkorte kopie is van je tekst!
In een presentatie kan je focussen op andere dingen en zullen bepaalde onderdelen van je tekst misschien niet aan bod komen.
Je presentatie moet gestructureerd zijn.
Denk dus, net als bij je tekst, op voorhand een verhaallijn uit en volg deze dan in je slides.
Overzichtslides zijn een handig hulpmiddel om deze verhaallijn te verduidelijken aan je publiek.
Prop nooit te veel informatie in \'e\'en slide.
Niet alleen maakt dit de slide onduidelijk, maar het zorgt er ook vaak voor dat het lettertype zo klein wordt dat de tekst niet leesbaar is vanop een afstand.


\subsection{De voorbereiding}

Een goede voorbereiding is cruciaal om een goede presentatie te geven.
Schenk daarom zeker voldoende aandacht aan de richtlijnen die jullie kunnen vinden in het beoordelingsformulier voor mondeling rapporteren.\footnote{\url{https://eng.kuleuven.be/studeren/engineering-essentials/rapporteren/mondeling}}
Daarenboven zijn de volgende aandachtspunten zeker interessant!


\paragraph{Verdeel de slides}

Aangezien er meerdere sprekers per presentatie zijn, zal je op voorhand onderling moeten afspreken wie welk deel van de presentatie voor zich neemt.
Zorg ervoor dat deze verdeling eerlijk is, elke spreker zal ongeveer evenveel tijd gesproken moeten hebben.
Indien \'e\'en van de sprekers opmerkelijk minder (of meer) presenteert dan de anderen, dan speelt dat niet in jullie voordeel.


\paragraph{Oefening baart kunst}

Om je presentatie vlot te kunnen brengen, zal je ze op voorhand moeten inoefenen.
Hoeveel je precies moet oefenen hangt af van persoon tot persoon af, dit zal je voor jezelf moeten bepalen.
Iemand die zijn presentatie goed ingeoefend heeft zal ook veel minder problemen ervaren tijdens de presentatie.
Stress voor een presentatie kan er namelijk voor zorgen dat de spreker zijn draad verliest of niet meer op bepaalde woorden kan komen.


\paragraph{Time management}

Zorg ervoor dat je niet over tijd gaat met je presentatie!
Groepjes die over tijd gaan, zullen abrupt worden afgebroken.
Het spreekt voor zich dat dit geen goede indruk nalaat.
Hou ook rekening met het feit dat de timing in veel gevallen anders zal zijn bij het inoefenen dan bij de daadwerkelijke voordracht.
Stress kan ervoor zorgen dat de spreker te snel (of traag) vooruit gaat, wat uiteindelijk ervoor kan zorgen dat de presentatie te lang (of kort) is.
Indien \'e\'en van de sprekers tijdens de presentatie te veel of te weinig tijd gebruikt, is het aan de volgende sprekers om dit te corrigeren.
Je zorgt er dus best voor dat vlotte sprekers het einde van de presentatie verzorgen.
Denk op voorhand eventueel al na welke onderdelen je kan overslaan indien er te weinig tijd zou zijn, of over welke onderdelen je eventueel iets langer zou kunnen spreken.
Tijdens de presentatie kan een klokje dat zichtbaar is voor de spreker handig zijn om de tijd in de gaten te houden.


\subsection{De voordracht}

Zelfs met prachtige slides en een goede voorbereiding ben je nog niet 100\% zeker dat de presentatie goed overkomt bij het publiek.
Hier zijn nog enkele dingen waar je best op kan letten tijdens je voordracht!


\paragraph{Lichaamstaal}

Je lichaamstaal kan boekdelen spreken.
Een spreker die constant naar zijn voeten kijkt komt heel onzelfzeker en nerveus over.
Langs de andere kant moet de spreker ook niet nonchalant overkomen.
Interageer met het publiek door ernaar te kijken! Dat wil natuurlijk niet zeggen dat je mensen moet aanstaren, maar je blik moet ongeveer 90\% van de tijd op de zaal gericht zijn.
Dit zorgt er ook voor dat je niet gewoon alles van de slides afleest.
Een humorvolle presentatie mag (en zal ook door het publiek bijzonder geapprecieerd worden), maar dan moet de humor natuurlijk wel correct gebracht worden.


\paragraph{Presentatie-technisch}

Spendeer genoeg tijd aan elke slide.
Een overzichtslide die een halve seconde getoond wordt is niet de moeite.
Indien je merkt dat je zo weinig over een slide kan zeggen, dan moet je deze misschien gewoonweg verwijderen.
Kleed het einde van de presentatie wat in met een conclusie zodat dit niet te abrupt is.
Vergeet het publiek niet te bedanken voor hun aandacht!


\flushright{}
Yolande Berbers\\
Gertjan Franken\\
Martijn Sauwens\\
Neline van Ginkel\\
Jan Vermaelen\\
Hans Winderix\\

\end{document}
